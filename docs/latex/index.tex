\hypertarget{index_overview_sec}{}\section{Overview}\label{index_overview_sec}
The Mol\-S\-S\-I Driver Interface (M\-D\-I) library enables codes to interoperate via the M\-D\-I.\hypertarget{index_source_sec}{}\section{Source Code}\label{index_source_sec}
The source code of the M\-D\-I library is available at Git\-Hub at \href{https://github.com/MolSSI/molssi_driver_interface}{\tt https\-://github.\-com/\-Mol\-S\-S\-I/molssi\-\_\-driver\-\_\-interface}\hypertarget{index_commands_sec}{}\section{Commands}\label{index_commands_sec}
The following is a list of commands that are officially part of the M\-D\-I standard.\hypertarget{index_send_name}{}\subsection{$<$\-N\-A\-M\-E}\label{index_send_name}
The engine returns a string of length {\ttfamily M\-D\-I\-\_\-\-N\-A\-M\-E\-\_\-\-L\-E\-N\-G\-T\-H} that corresponds to the argument of {\ttfamily -\/name} in the M\-D\-I initialization options. This argument allows a driver to identify the purpose of connected engine codes within the simulation. For example, a particular Q\-M/\-M\-M driver might require a connection with a single M\-M code and a single Q\-M code, with the expected name of the M\-M code being \char`\"{}\-M\-M\char`\"{} and the expected name of the Q\-M code being \char`\"{}\-Q\-M\char`\"{}. After initializing M\-D\-I and accepting communicators to the engines, the driver can use this command to identify which of the engines is the M\-M code and which is the Q\-M code.\hypertarget{index_md_init}{}\subsection{M\-D\-\_\-\-I\-N\-I\-T}\label{index_md_init}
The engine performs any initialization operations that are necessary before an M\-D simulation can be time propagated. 